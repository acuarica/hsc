\begin{abstract}
Supercompilation is a program transformation technique aimed at reducing superfluous computations.
Although being a single transformation technique, it can be used for multiple applications, \eg, program optimization, theorem proving, or logic programming.

But to date, there is no mainstream Supercompiler being used in practice, neither in academic or commercial projects.
This is because Supercompilation has some shortcomings, \eg, code-explosion, or time-consumption, for the aforementioned applications.
Nevertheless, we believe that some specific domains can benefit from Supercompilation despite its shorcomings.

In this proposal, we study how actually Supercompilation is well-suited for a wide-range of domains.
We will explore different specific domains where we believe that Supercompilation might have better results than the current state-of-art.
Our main goal is to make Supercompilation practical, by finding a \emph{``killer application''} for it.
\end{abstract}
